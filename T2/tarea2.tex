\documentclass[twoside]{tareas}

\usepackage[spanish]{babel}
\usepackage[utf8]{inputenc}

\usepackage{amssymb}
\usepackage{amsfonts}
\usepackage{amsmath}
\usepackage{amsthm}
\usepackage{latexsym}
\usepackage{algorithmicx,algpseudocode,  algorithm}


\newcommand{\gen}{\mathsf{Gen}}
\newcommand{\enc}{\mathsf{Enc}}
\newcommand{\dec}{\mathsf{Dec}}
\newcommand{\sign}{\mathsf{Sign}}
\newcommand{\vrfy}{\mathsf{Vrfy}}
\newcommand{\mac}{\mathsf{Mac}}

\newcommand{\la}{\langle}
\newcommand{\ra}{\rangle}
\newcommand{\tuple}[1]{\la#1\ra}
\newcommand{\calA}{\mathcal{A}}
\newcommand{\calB}{\mathcal{B}}
\newcommand{\calC}{\mathcal{C}}
\newcommand{\calD}{\mathcal{D}}
\newcommand{\calE}{\mathcal{E}}
\newcommand{\calG}{\mathcal{G}}
\newcommand{\calH}{\mathcal{H}}
\newcommand{\calI}{\mathcal{I}}
\newcommand{\calJ}{\mathcal{J}}
\newcommand{\calK}{\mathcal{K}}
\newcommand{\calL}{\mathcal{L}}
\newcommand{\calM}{\mathcal{M}}
\newcommand{\calN}{\mathcal{N}}
\newcommand{\calO}{\mathcal{O}}
\newcommand{\calP}{\mathcal{P}}
\newcommand{\calQ}{\mathcal{Q}}
\newcommand{\calR}{\mathcal{R}}
\newcommand{\calS}{\mathcal{S}}
\newcommand{\calT}{\mathcal{T}}
\newcommand{\calU}{\mathcal{U}}
\newcommand{\calV}{\mathcal{V}}
\newcommand{\calW}{\mathcal{W}}
\newcommand{\calX}{\mathcal{X}}
\newcommand{\calY}{\mathcal{Y}}
\newcommand{\calZ}{\mathcal{Z}}
\newcommand{\getunif}{\stackrel{\$}{\gets}}

\nombre{Raimundo Herrera}
\correo{rjherrera@uc.cl}
\colab{}
\curso{IIC3253: Criptografía y Seguridad Computacional}
\tareaNo{2}

\begin{document}
\maketitle

\section*{Problema 1}

Determine si las siguientes funciones $G'$ son generadores pseudo-aleatorios.

\begin{enumerate}
    \item {\boldmath$G': \{0,1\}^{2\lambda} \rightarrow \{0,1\}^{n}, G'(x = x_1 x_2 ...x_{2\lambda}) = G(x_1 x_2 ...x_{\lambda})$}

    $G'$ si es $PRG$. Para demostrarlo, lo haré por reducción. Asumamos que $G'$ no es pseudo-aleatorio, si esto es así, entonces existe un distinguidor $D'$ tal que

    $$\Pr[D'(G'(U_{2\lambda})) = 1] - \Pr[D'(U_{2\lambda}) = 1] > \text{negl}(2\lambda)$$

    Se puede construir un distinguidor $D(z)$ para $G$ de la siguiente forma

    \begin{itemize}
        \item $b \leftarrow D'(z)$
        \item output $b$
    \end{itemize}

    Como ambos distinguidores $G$ y $G'$ reciben como input elementos de $n$ bits, su funcionamiento es análogo. Basta demostrar entonces que efectivamente son equivalentes, esto es, que este último distinguidor especificado distingue $G$.

    Evidentemente sabemos que
    $$\Pr[D'(U_\lambda) = 1] = \Pr[D(U_\lambda) = 1]$$

    Por otro lado, notemos que el input de $G'$ viene de $U_{2\lambda}$, y que el de $G$ viene de $U_\lambda$. Como son ambas distribuciones uniformes, se puede mirar el input de $G'$ como $K||K'$. De este modo, independiente del valor de $K'$ se tiene, por construcción de $G'$ que
    $$G'(K||K') = G(K)$$

    Así, es inmediato que
    $$\Pr_{K,K' \sim U_\lambda}[D'(G'(K||K')) = 1] = \Pr_{K \sim U_\lambda}[D(G(K)) = 1]$$
    Ya que los distinguidores son equivalentes y $G'(K||K') = G(K)$.

    De este modo, como ambas probabilideades son iguales, se tiene que
    $$\Pr[D'(U_{2\lambda}) = 1] - \Pr[D'(U_{2\lambda}) = 1] = \Pr[D(U_\lambda) = 1] - \Pr[D(U_\lambda) = 1]$$

    Y como el lado de la izquierda por lo que asumimos en un principio es no negligible, entonces el lado derecho también. De este modo, se concluye que si $G'$ no es $PRG$ entonces $G$ tampoco lo es, lo que contradice lo enunciado inicialmente. De este modo, por contrapositivo se demuestra que $G'$ si es $PRG$.
    \begin{flushright} $\blacksquare$ \end{flushright}

    \item {\boldmath$G': \{0,1\}^{\lambda} \rightarrow \{0,1\}^{2n}, G'(x) = G(x)||G(x+1)$}

    $G'$ no es $PRG$. Para mostrarlo, construiré $G$ de forma que, siendo $G$ efectivamente $PRG$, se pueda generar un distinguidor $D'$ para $G'$, que muestre que no es $PRG$.

    Sea $G'': \{0,1\}^{\lambda - 1} \rightarrow \{0,1\}^{n-1}$ un $PRG$, definamos $G$ de la siguiente manera
    $$G: \{0,1\}^{\lambda} \rightarrow \{0,1\}^{n}, G(x_1 x_2 ...x_{\lambda}) = G''(x_1 x_2 ...x_{\lambda - 1}) || x_{\lambda}$$

    Es fácil ver que el $G$ propuesto es pseudo-aleatorio ya que los primeros $n-1$ bits son generados por $G''$ que es $PRG$ y el último bit $x_{\lambda}$ es independiente de los primeros. Realizando reducción y por contrapositivo se puede demostrar, de este modo, demostraré que, si $G$ no es pseudo-aleatorio entonces $G''$ tampoco lo es.

    Si $G$ no es pseudo-aleatorio, entonces existe $D^*$ tal que distingue $G$.

    Construyamos $D^{**}(z=z_1...z_{\lambda-1})$ tal que

    \begin{itemize}
        \item $u \sim U_1$
        \item $x = z || u$
        \item $b \leftarrow D^*(x)$
        \item output $b$
    \end{itemize}

    Ese distinguidor toma su input $z$, le agrega un bit aleatorio como último elemento y este nuevo string lo utiliza en el distinguidor que asumimos existía para $G$, luego entregando el output que este de. De este modo, si existe un distinguidor para $G$, entonces existe un distinguidor para $G''$, de modo que, $G''$ no podría ser $PRG$. Y como lo es, entonces por el argumento expuesto, $G$ tiene que serlo.

    Continuando con el contraejemplo, recapitulemos. Tenemos el siguiente $G(x) = G''(x_1 x_2 ...x_{\lambda - 1}) || x_{\lambda}$ que es pseudo-aleatorio. Lo que sigue es mostrar que usando ese $G$ en la definición de $G'$, este último pasa a ser distinguible.

    Tomemos $G'$ en términos del nuevo $G$, y sea $y=x+1$
    \begin{align*}
        G'(x) &= G(x)||G(y)\\
        G'(x) &= G''(x_1 x_2 ...x_{\lambda - 1}) || x_{\lambda} || G''(y_1 y_2 ...y_{\lambda - 1}) || y_{\lambda}
    \end{align*}
    Sabemos que $x$ es un string binario, y en caso de que el número que represente sea par, al sumarle 1, el resultado, es decir $y$, solo difiere en el último bit, pasando de 0 a 1. De este modo basta construir un distinguidor que verifique que los $n - 1$ primeros bits son iguales a los $n-1$ bits de la segunda mitad (ya que $G''$ habría recibido el mismo input de $\lambda - 1$ bits para cada uno, por lo tanto entrega el mismo output) y que el enésimo bit sea 0 y el último 1. Así, sea $D'(z)$ el siguiente distinguidor

    \begin{itemize}
        \item $a = z_1 ...z_{n-1}$
        \item $b = z_{n+1} ...z_{2n-1}$
        \item si $a=b$ y $z_n = 0$ y $z_{2n}=1$ output 1
        \item en otro caso output $0$
    \end{itemize}

    Este distinguidor compara los primeros dígitos de cada mitad del output de $G'$, de modo de poder comprobar la igualdad. Ahora, con este distinguidor el análisis de probabilidades para el caso en que efectivamente proviene de un generador es el siguiente:
    \begin{align*}
        \Pr[D(G'(U_\lambda)) = 1] &= \Pr[D(G'(U_\lambda)) = 1\ | \ U_\lambda \text{ par}] \cdot \Pr[U_\lambda \text{ par}]\\
        &+ \Pr[D(G'(U_\lambda)) = 1\ | \ U_\lambda \text{ impar}] \cdot \Pr[U_\lambda \text{ impar}]\\
        &= \Pr[D(G'(U_\lambda)) = 1\ | \ U_\lambda \text{ par}] \cdot 1/2\\
        &+ \Pr[D(G'(U_\lambda)) = 1\ | \ U_\lambda \text{ impar}] \cdot 1/2\\
        &= 1 \cdot 1/2 + 0 \cdot 1/2\\
        &= 1/2
    \end{align*}
    Ahora, la segunda probabilidad, es decir, si el input proviene de un string aleatorio:
    \begin{align*}
        \Pr[D(U_\lambda) = 1] &= 2^{-(n + 1)}
    \end{align*}
    Esto porque, del universo de todos los strings binarios de largo $2n$, un cuarto de ellos tienen el dígito de al medio igual a 0 y el del final igual a 1. Además, la cantidad de strings cuyos bits de la primera mitad son iguales a los de la segunda (sin considerar el de al medio y el final), corresponden a $2^{n+1}$. Así la cantidad de strings que cumplen ambas condiciones son $2^{n - 1}$. Por lo que, de un universo de $2^{2n}$ strings, se obtiene la probabilidad explicitada.

    Finalmente, restando ambas probabilidades se tiene:
    \begin{align*}
        \Pr[D(G'(U_\lambda)) = 1] - \Pr[D(U_\lambda) = 1] &= 1/2 - 2^{-(n + 1)}
    \end{align*}
    Lo que evidentemente es no negligible. Demostrandose que con dicho distinguidor, se puede distinguir con una probabilidad no neglibible, por lo que $G'$ no es $PRG$.

    \item {\boldmath$G': \{0,1\}^{\lambda} \rightarrow \{0,1\}^{n+1}, G'(x) = G(x)||x_1$}

    $G'$ no es $PRG$. Con un procedimiento similar, sea $G'': \{0,1\}^{\lambda - 1} \rightarrow \{0,1\}^{n-1}$ un $PRG$, definamos $G$ de la siguiente manera
    $$G: \{0,1\}^{\lambda} \rightarrow \{0,1\}^{n}, G(x_1 x_2 ...x_{\lambda}) = x_1 || G''(x_2 x_3...x_{\lambda})$$

    Evidentemente $G$ es un $PRG$. La demostración por contrapositivo de que efectivamente es pseudo-aleatorio es idéntica a la de (1.2), solo que concatenando el bit aleatorio al principio del string, es decir, en el distinguidor hay que especificar $x = u||z$.

    De este modo, hay que mostrar que al usar este $G$ que es $PRG$, se puede fabricar un distinguidor que quiebra $G'$. Tomemos entonces $G'$ en términos del nuevo $G$ recién definido:
     \begin{align*}
        G'(x) &= G(x)||x_1\\
        G'(x) &= x_1 || G''(x_2 x_3...x_{\lambda}) || x_1
    \end{align*}

    De este modo, sea $D'(z)$ el siguiente distinguidor

    \begin{itemize}
        \item si $z_1 = z_{n}$ output 1
        \item en otro caso output 0
    \end{itemize}

    Con este distinguidor el análisis de probabilidades se vuelve trivial.
    \begin{align*}
        \Pr[D(G'(U_\lambda)) = 1] &= 1
    \end{align*}
    Ya que siempre que venga del generador tendrá los bits inicial y final iguales. Por otro lado, si tomamos la otra probabilidad tenemos
    \begin{align*}
        \Pr[D(U_\lambda) = 1] &= 1/2
    \end{align*}
    Ya que la probabilidad de que un string binario tenga los bits inicial y final iguales es de 1/2 (0-0 y 1-1). Teniendo finalmente que
    \begin{align*}
        \Pr[D(G'(U_\lambda)) = 1] - \Pr[D(U_\lambda) = 1] &= 1 - 1/2\\
        &= 1/2
    \end{align*}
    Lo que no es neglibible, teniendo entonces que $G'$ no es $PRG$.
\end{enumerate}

\section*{Problema 2}

Determine si las siguientes funciones $F'$ son también funciones pseudo-aleatorios.

\begin{enumerate}
    \item {\boldmath$F': \{0, 1\}^{\lambda} \times \{0, 1\}^{\lambda-1} \rightarrow \{0, 1\}^{2\lambda}, F'(k, x) = F (k, 0||x)||F (k, 1||x)$}

    $F'$ sí es PRF. Por propiedades de las funciones pseudo-aleatorias, sea $f$ una de ellas, para $i \neq j$, $f(i)$ es independiente de $f(j)$. En este caso se puede tener en cuenta que $0||x$ y $1||x$ nunca serán iguales, y de este modo, para $x_1$ y $x_2$ distintos, $0||x_1$, $1||x_1$, $0||x_2$, $1||x_2$ son todos distintos.

    Así, los outputs de $F$ para cada caso serán independientes, por lo tanto no podré fabricar nunca un distinguidor que gane con probabilidad mayor a negligible, ya que nunca podrá igualar partes de los resultados de $F_k$ con una certeza mayor que si viniera de una distribución aleatoria.

    La demostración formal es por contrapositivo y se desprende inmediatamente de construir el oráculo, pero no alcancé a incluirla.

    \item {\boldmath$F': \{0, 1\}^{\lambda} \times \{0, 1\}^{\lambda-1} \rightarrow \{0, 1\}^{2\lambda}, F'(k, x) = F (k, 0||x)||F (k, x||1)$}

    $F'$ no es PRF. Para mostrarlo construiré un distinguidor que sea capaz de distinguir a $F'$ de una función aleatoria en tiempo polinomial. Sea $D_F'(1^{1-\lambda}$ con acceso al oráculo $\mathcal{O}$ construido de la siguiente forma

    \begin{itemize}
        \item $a = \mathcal{O}(0^{\lambda - 1})$
        \item $b = \mathcal{O}(0^{\lambda - 2} || 1)$
        \item si $a_{\lambda + 1}...a{2\lambda} = b_{1}...b{\lambda}$ output 1
        \item en otro caso output 0
    \end{itemize}

    Con este distinguidor, intuitivamente se ve que se puede lograr distinguir cuando la segunda mitad de la primera llamada al oráculo, es igual a la primera mitad de la segunda llamada al oráculo.

    Haciendo el análisis de probabilidades, se tiene que para cualquier $k$ con el distinguidor anterior, la condición expresada se va a cumplir, ya que sucederá que la segunda mitad de $a$ será $F_k(0^{\lambda - 1} || 1)$ y la primera mitad de $b$ será $F_k(0^{\lambda - 1} || 1)$, también, por lo que la probabilidad es
    $$\Pr_{k \sim U_\lambda}[D_F^{F_k}(1^{\lambda-1}) = 1] = 1$$

    Por otro lado, la probabilidad cuando se tiene un $f$ cualquiera como oráculo, corresponde a la probabilidad de que las dos mitades sean iguales, y el análisis es el mismo que para una distribución uniforme de largo $\lambda$, por lo que
    $$\Pr_{f \sim \mathcal{F}_\lambda}[D_F^{f}(1_\lambda) = 1] = 1/2^{\lambda}$$

    De este modo la probabilidad siguiente se vuelve no negligible, de modo que no es $PRF$.
    \begin{align*}
        \Pr_{k \sim U_\lambda}[D_F^{F_k}(1^{\lambda-1}) = 1] - \Pr_{f \sim \mathcal{F}_\lambda}[D_F^{f}(1^\lambda) = 1] &= 1 - 1/2^{\lambda}\\
        &> \text{negl}(\lambda - 1)
    \end{align*}

    \item {\boldmath$F': \{0, 1\}^{\lambda} \times \{0, 1\}^{\lambda} \rightarrow \{0, 1\}^{\lambda}, F'(k, x) = F (x,k)$}

    $F'$ no es PRF. Para mostrarlo consideraré un $F$ que si es $PRF$ para el cual, utilizandolo en $F'$ se puede construir un distinguidor.

    Sea $F''$ una $PRF$, se define $F$ de la siguiente forma:

    $$F = \begin{cases} F''(k, x) & \text{ si } k \neq 0^{\lambda} \\ 0^{\lambda} & \text{ si } k = 0^{\lambda} \end{cases}$$

    Es posible demostrar por contrapositivo que si $F''$ es $PRF$ entonces $F$ también lo es. Es sencillo, sin embargo, notar que $F$ es $PRF$ puesto que solo difiere en un elemento, $0^{\lambda}$. Desde el punto de vista de las probabilidades, como ese elemento es solamente 1 de un universo de $2^{\lambda}$, es negligible la probabilidad de distinguir, ya que el caso se presenta una cantidad negligible de veces.

    Volviendo al análisis, usando este $F$ propuesto para $F'$, se puede construir el siguiente distinguidor $D_F$ con acceso al oráculo $\mathcal{O}$.

    \begin{itemize}
        \item $a = \mathcal{O}(0^{\lambda})$
        \item si $a = 0^{\lambda}$ output 1
        \item en otro caso output 0
    \end{itemize}

    Haciendo el análisis de probabilidades, se tiene que, para el caso en que $\mathcal{O}$ se comporta como $F'$, como le damos el mensaje $0^{\lambda}$, este será la llave para nuestro $F$, entonces, el output será siempre el mismo $0^{\lambda}$, por lo que
    $$\Pr_{k \sim U_\lambda}[D_F^{F'_k}(1^{\lambda}) = 1] = 1$$

    Por el otro lado, si el oráculo se comporta como $f$, es decir, se comportaría como una distribución uniforme, la probabilidad de que el oráculo arroje $0^{\lambda}$ es 1 caso de entre el total de casos posibles $2^{\lambda}$, esto es
    $$\Pr_{f \sim \mathcal{F}_\lambda}[D_F^{f}(1^\lambda) = 1] = 1/2^{\lambda}$$

    Finalmente, con esto, se tiene que
    \begin{align*}
        \Pr_{k \sim U_\lambda}[D_F^{F'_k}(1^{\lambda}) = 1] - \Pr_{f \sim \mathcal{F}_\lambda}[D_F^{f}(1^\lambda) = 1] &= 1 - 1/2^{\lambda}\\
        &> \text{negl}(\lambda)
    \end{align*}



\end{enumerate}

\section*{Problema 3}

\textbf{PRFs} {\boldmath$\implies$} \textbf{PRGs}

Similarmente al argumento utilizado siempre, demostrar lo pedido equivale a, por contrapositivo, demostrar que si el $G$ propuesto no es $PRG$, entonces $F$ no es $PRF$.

Pero antes de realizar la demostración para la pseudo-aleatoriedad, cabe señalar que la expansión $\lambda \cdot \ell$ es inmediata. Como $F_k$ es una función cuyo output es de largo $\lambda$ para todo input, y en la construcción de $G$, se concatena $\ell$ veces $F_k$, entonces trivialmente la expansión de $G$ es $\lambda \cdot \ell$

Volviendo, si $G$ no es pseudo-aleatorio, entonces existe un $D_G$ tal que
$$\Pr[D_G(G(U_\lambda)) = 1] - \Pr[D_G(U_\lambda) = 1] > \text{negl}(\lambda)$$

Lo que se busca es poder crear un distinguidor $D_F$ para $F$ tal que
$$\Pr_{k \sim U_\lambda}[D_F^{F_k}(1^\lambda) = 1] - \Pr_{f \sim \mathcal{F}_\lambda}[D_F^{f}(1^\lambda) = 1] > \text{negl}(\lambda)$$
Donde tanto $k$ como $f$ son elegidos uniformemente de sus respectivos conjuntos. Para construir $D_F$, se tiene el oráculo $\mathcal{O}$ que responde en tiempo constante cuando se le solicita un input.

Para el distinguidor, como puede consultar cuantas veces quiera al oráculo, sea
$$y_i = \mathcal{O}(i) \ \forall i \in \{0, ..., \ell\}$$
Armando entonces la siguiente expresión para $y$:
$$y = y_1||y_2||...||y_\ell$$

Ahora, el distinguidor tras tener construido $y$, llama al distinguidor que ya sabemos que existe, $D_G$, de modo que su procedimiento es
\begin{itemize}
    \item $b \leftarrow D_G(y)$
    \item output $b$
\end{itemize}

El análisis es el siguiente, si el oráculo es $F_k(\cdot)$, entonces por construcción de $G$, se tiene que es exactamente equivalente a $y$, por lo que el distinguidor gana con probabilidad no negligible. En caso contrario, esto es, el oráculo es $f(\cdot)$, entonces gana con probabilidad negligible.

Desde el punto de vista de las probabilidades, se tiene que, como $y$ es justamente la construcción de $G$,
$$\Pr_{k \sim U_\lambda}[D_F^{F_k}(1^\lambda) = 1] = \Pr[D_G(G(U_\lambda)) = 1]$$

Por otro lado, las funciones aleatorias $f$ elegidas, son independientes entre si, esto es $f(i)$ es elegida independientemente de $f(j)$ para todo par $i\neq j$. De este modo, $y$ corresponde a escoger aleatoriamente un número de uns distribución uniforme, ya que cada función se comporta como tal. De esto se tiene que
$$\Pr_{f \sim \mathcal{F}_\lambda}[D_F^{f}(1^\lambda) = 1] = \Pr[D_G(U_\lambda) = 1]$$

Entonces, finalmente se tiene que
$$\Pr[D_G(G(U_\lambda)) = 1] - \Pr[D_G(U_\lambda) = 1] = \Pr_{k \sim U_\lambda}[D_F^{F_k}(1^\lambda) = 1] - \Pr_{f \sim \mathcal{F}_\lambda}[D_F^{f}(1^\lambda) = 1] > \text{negl}(\lambda)$$

Como el distinguidor para $F$ se reduce a utilizar el distinguidor para $G$ con los mismos resultados de este último, entonces se concluye que existe un $D_F$ construido como se explicitó, tal que $F$ no es una $PRF$ si $G$ no es $PRG$.

Por lo tanto, por contrapositivo, $G$ ha de ser $PRG$, ya que $F$ efectivamente es $PRF$.
\begin{flushright} $\blacksquare$ \end{flushright}



\end{document}
