\documentclass[twoside]{tareas}

\usepackage[spanish]{babel}
\usepackage[utf8]{inputenc}

\usepackage{amssymb}
\usepackage{amsfonts}
\usepackage{amsmath}
\usepackage{amsthm}
\usepackage{latexsym}
\usepackage{algorithmicx,algpseudocode,  algorithm}


\newcommand{\gen}{\mathsf{Gen}}
\newcommand{\enc}{\mathsf{Enc}}
\newcommand{\dec}{\mathsf{Dec}}
\newcommand{\sign}{\mathsf{Sign}}
\newcommand{\vrfy}{\mathsf{Vrfy}}
\newcommand{\mac}{\mathsf{Mac}}

\newcommand{\la}{\langle}
\newcommand{\ra}{\rangle}
\newcommand{\tuple}[1]{\la#1\ra}
\newcommand{\calA}{\mathcal{A}}
\newcommand{\calB}{\mathcal{B}}
\newcommand{\calC}{\mathcal{C}}
\newcommand{\calD}{\mathcal{D}}
\newcommand{\calE}{\mathcal{E}}
\newcommand{\calG}{\mathcal{G}}
\newcommand{\calH}{\mathcal{H}}
\newcommand{\calI}{\mathcal{I}}
\newcommand{\calJ}{\mathcal{J}}
\newcommand{\calK}{\mathcal{K}}
\newcommand{\calL}{\mathcal{L}}
\newcommand{\calM}{\mathcal{M}}
\newcommand{\calN}{\mathcal{N}}
\newcommand{\calO}{\mathcal{O}}
\newcommand{\calP}{\mathcal{P}}
\newcommand{\calQ}{\mathcal{Q}}
\newcommand{\calR}{\mathcal{R}}
\newcommand{\calS}{\mathcal{S}}
\newcommand{\calT}{\mathcal{T}}
\newcommand{\calU}{\mathcal{U}}
\newcommand{\calV}{\mathcal{V}}
\newcommand{\calW}{\mathcal{W}}
\newcommand{\calX}{\mathcal{X}}
\newcommand{\calY}{\mathcal{Y}}
\newcommand{\calZ}{\mathcal{Z}}
\newcommand{\getunif}{\stackrel{\$}{\gets}}

\nombre{Raimundo Herrera}
\correo{rjherrera@uc.cl}
\colab{Thomas Reisenegger}
\curso{IIC3253: Criptografía y Seguridad Computacional}
\tareaNo{3}

\begin{document}
\maketitle

\section*{Problema 1}

Calcule a mano lo siguiente (puede ocupar el teorema del resto chino).

\begin{enumerate}
    \item \textbf{Los últimos 2 díıgitos de} {\boldmath$7^{35688}$}.

    Lo pedido equivale a encontrar el resultado de $7^{35688} \mod 100$. Además, por el \textit{hint} se tiene que
    $$\phi(100) = (2^1 \cdot 1) (5 ^ 1 \cdot 4) = 40.$$
    Por otro lado, sabemos $\forall a \in \mathbb{Z}^*_N$, $\forall x \in \mathbb{Z}$ se tiene que
    $$[a^{x} \mod N] = [a^{x \mod \phi(N)} \mod N]$$
    de modo que para el caso particular se tiene
    $$[7^{35688} \mod 100] = [7^{35688 \mod 40} \mod 100].$$

    Ahora, considerando que $[35688 \mod 40] = 8$ se tiene que
    \begin{align*}
        [7^{35688 \mod 40} \mod 100] &= [7^{8} \mod 100]\\
        &= [(7^4)^2 \mod 100]\\
        &= [(2401)^2 \mod 100]\\
        &= [(2401 \mod 100)^2 \mod 100]\\
        &= [(1)^2 \mod 100]\\
        &= 1\\
    \end{align*}

    De modo que finalmente los últimos dos dígitos de $7^{35688}$ son $01$.


    \item {\boldmath$[233^{230000022} \mod 35]$}.

    La descomposición prima de $35$ es $7\cdot 5$ y por teorema del resto chino se tiene que primero hay que representar $233$ como elemento en $\mathbb{Z}^*_7 \times \mathbb{Z}^*_5$, y esto es $(2, 3)$. De este modo,
    \begin{align*}
        [233^{230000022} \mod 35] &\leftrightarrow ([(2)^{230000022} \mod 7], [(3)^{230000022} \mod 5])\\
        &\leftrightarrow (2^{230000022} \mod 7, 3^{230000022} \mod 5)
    \end{align*}

    Por otro lado se tiene también que $230000022$ es divisible por 3 y par, ya que (1) la suma de sus dígitos es 9 y (2) termina en 2. De este modo, si $a = 230000022 / 3$ y $b = 230000022 / 2$ (notar que $b$ es impar), se tiene que
    \begin{align*}
        (2^{230000022} \mod 7, 3^{230000022} \mod 5) &= (2^{3\cdot a} \mod 7, 3^{2\cdot b} \mod 5)\\
        &= (8^{a} \mod 7, 9^{b} \mod 5)\\
        &= (1^{a} \mod 7, -1^{b} \mod 5)\\
        &= (1 \mod 7, -1 \mod 5)\\
        &= (1, -1)\\
        &= (1, 4).
    \end{align*}

    Ahora, para encontrar el número cuya representación en $\mathbb{Z}^*_7 \times \mathbb{Z}^*_5$ es $(1, 4)$ hay que primero encontrar $x$ e $y$ tales que $7x + 5y = 1$. Para esto, utilizando el algoritmo de Euclides extendido se tiene que $x = -2$ e $y = 3$.

    Con esto podemos obtener $1_7$ y $1_5$, que corresponden a $1_p = [y\cdot Q \mod N]$ y $1_q = [x\cdot P \mod N]$ respectivamente. Entonces
    \begin{align*}
        1_7 &= [3\cdot 5 \mod 35]\\&= 15\\
        1_5 &= [-2\cdot 7 \mod 35]\\&= 21.
    \end{align*}

    Así, podemos usar el último paso del algoritmo visto en clases que dice que el número buscado es $x_p \cdot 1_p + y_q \cdot 1_q \mod N$, en particular
    \begin{align*}
        x_7 \cdot 1_7 + y_5 \cdot 1_5 \mod 35 &= [1 \cdot 15 + 4 \cdot 21 \mod 35]\\
        &= [99 \mod 35]\\
        &= 29.
    \end{align*}

    Por lo que el resultado es $[233^{230000022} \mod 35] = 29$.

    \item {\boldmath$[46^{51} \mod 55]$}.

    Por el mismo  \textit{hint} de la parte (1) se tiene que $\phi(55) = (5^0 \cdot 4)(11^0 \cdot 10) = 40$. Así, podemos usar que
    \begin{align*}
        [46^{51} \mod 55] &= [46^{51 \mod 40} \mod 55]\\
        &= [46^{11} \mod 55]
    \end{align*}

    Por teorema del resto chino, usando que la descomposición prima de $55$ es $5 \cdot 11$ tenemos que al expresar $55$ como elemento de $\mathbb{Z}^*_5 \times \mathbb{Z}^*_{11}$ se obtiene $(1, 2)$, así
    \begin{align*}
        [46^{11} \mod 55] &\leftrightarrow ([(1)^{11} \mod 5], [(2)^{11} \mod 11])\\
        &\leftrightarrow (1 \mod 5, 2^{11} \mod 11)
    \end{align*}

    Ahora, $[2^{11} \mod 11]$ se puede expresar como $[(2^5)(2^5)(2) \mod 11]$ y por propiedades de la aritmética modular, eso es quivalente a $[[2^5 \mod 11][2^5 \mod 11][2 \mod 11] \mod 11]$, lo que a su vez, como $[32 \mod 11] = -1$, termina siendo equivalente a $[2 \mod 11]$. Con esto podemos decir finalmente que en $\mathbb{Z}^*_5 \times \mathbb{Z}^*_{11}$ se tiene lo siguiente
    $$(1 \mod 5, 2^{11} \mod 11) = (1, 2).$$

    Así procediendo análogamente para encontrar a qué numero equivale el $(1, 2)$, debemos encontrar $x$ e $y$ tales que $5x + 11y = 1$. Utilizando el algoritmo extendido de Euclides se tiene que $x = -2$ e $y = 1$.

    Con esto podemos obtener $1_5$ y $1_{11}$, que corresponden a $1_p = [y\cdot Q \mod N]$ y $1_q = [x\cdot P \mod N]$ respectivamente. Por lo tanto
    \begin{align*}
        1_5 &= [1\cdot 11 \mod 55]\\&= 11\\
        1_{11} &= [-2\cdot 5 \mod 55]\\&= 45.
    \end{align*}

    Ahora, usando último paso del algoritmo hay que retornar $x_p \cdot 1_p + y_q \cdot 1_q \mod N$, en particular
    \begin{align*}
        x_5 \cdot 1_5 + y_11 \cdot 1_{11} \mod 55 &= [1 \cdot 11 + 2 \cdot 45 \mod 55]\\
        &= [101 \mod 55]\\
        &= 46.
    \end{align*}

    Por lo que el resultado es $[46^{51} \mod 55] = 46$.
\end{enumerate}

\section*{Problema 2}

Sea $\mathbb{G}$ es un grupo cíclico de orden $n$ y generador $g$.

\begin{enumerate}
    \item \textbf{Demuestre que {\boldmath$\mathbb{Z}_n \textbf{ y } \mathbb{G}$} son isomorfos.}

    Para demostrar lo pedido, basta con encontrar una función biyectiva o permutación, entre $\mathbb{Z}_n$ y $\mathbb{G}$, y que esta prerserve estructura.

    De este modo, consideremos que los elementos del grupo se expresan como $g^i$ con $i \in \mathbb{Z}_n$. Es inmediato ver que la cardinalidad de $\mathbb{G}$ es igual a la de $\mathbb{Z}_n$, por ende una biyección es posible. Sea $f: \mathbb{Z}_n \rightarrow \mathbb{G}$, tal que
    $$f(x) = g^x$$

    La inversa de esta función es inmediata, abusando de notación para que el argumento de la función inversa sea un elemento de $\mathbb{G}$:
    $$f^{-1}(g^x) = x$$

    Por construcción dicha función es biyectiva. Ahora para fijarse en si preserva o no estructura, se debe cumplir que
    $$\forall g_i, g_j \in \mathbb{G}, \ f(g_i \circ_{\mathbb{G}} g_j) = f(g_i) \circ_{\mathbb{Z}_n} f(g_j)$$

    En este caso, se tiene que $\circ_{\mathbb{Z}_n}$, esto es, la operación de $\mathbb{Z}_n$ es la suma módulo $n$, y la operación del grupo $\mathbb{G}$ se puede considerar en notación multiplicativa. Como al hablar de índices en el grupo, se tiene que $g_i$ corresponde al $i$-ésimo elemento y este corresponde a su vez al elemento $g^i$, se debe cumplir que
    $$f(g^i * g^j) = f(g^i) + f(g^j) \mod n$$

    Es inmediato ver que se cumple ya que
    \begin{align*}
        f(g^i * g^j) &= f(g^{i + j})\\
        &=f(g^{i + j \mod n})\\
        &=i + j \mod n
    \end{align*}
    Así, $f$ es un isomorfismo entre $\mathbb{Z}_n$ y $\mathbb{G}$, por lo que son isomorfos. \begin{flushright} $\blacksquare$ \end{flushright}
    \pagebreak

    \item \textbf{Asumiendo que {\boldmath$n = p \cdot q$}, en donde {\boldmath$p \textbf{ y } q$} son primos distintos ¿Cuantos generadores tiene {\boldmath$\mathbb{G}$}? Demuestre su resultado.}

    Los grupos cíclicos tienen un generador $g_*$ tal que $\mathbb{G} = \{g^0_*, g^1_*, ..., g^{q-1}_*\}$ donde $q$ es el mínimo entero positivo tal que $g^q_* = 1$.

    % Si se toma dicho $g_*$, como genera todos los elementos del grupo, se tiene que para algún $i$, $g^i_*$ generará el generador $g$ dado.

    Primero demostraré que todos los elementos del grupo $g^k$ donde $\texttt{mcd}(k, n) = 1$, o equivalentemente $k$ y $n$ son primos relativos, generan el grupo. Para esto mostraré que un elemento $g^x$ genera el grupo si y solo si $x$ es coprimo con $n$ (orden del grupo).

    Por propiedad de los coprimos existen $a$ y $b$ tal que $a\cdot x + b\cdot n = 1$, entonces se puede hacer lo siguiente, para todo $y \in \mathbb{Z}$
    \begin{align*}
        g^y &= g^{y \cdot 1}\\
        &=g^{y(ax + bn)}\\
        &=g^{yax + ybn)}\\
        &=g^{yax}\cdot g^{ybn}\\
        &=g^{yax}\text{,  ya que } g^{ybn} = 1\\
        &=(g^x)^{ya}
    \end{align*}

    Por lo que como era para todo $y$, $g^x$ genera el grupo. Ahora para el si y solo si falta demostrar que es suficiente, es decir la implicancia para el otro lado, esto es, si $g^x$ es generador, genera el grupo.

    Tomemos $g^x$ como el generador del grupo, entonces existe un $l$ tal que $(g^x)^l = g$, ya que $g$ es un elemento del grupo y se puede generar con el generador. Si esto ocurre, entonces en particular se tiene que dar que $x\cdot l = 1 \mod n$, por lo visto en clases, pero es evidente, ya que para que genere a $g = g^1$ se necesita que el producto del exponente en módulo $n$ sea $1$. De este modo, por propiedades de la congruencia $\mod n$, se tiene que también existen $a$ y $b$ tal que $a\cdot x + b\cdot n = 1$, que es justamente la definición de coprimos para $x$ y $n$.

    Una vez probado lo anterior, esto es, $g^x$ genera el grupo si y solo si $x$ es coprimo con $n$, se tiene que todos los generadores del grupo son de orden coprimo. Esto implica que la cantidad de generadores es igual a la cantidad de elementos coprimos con $n$ en $\{0,...,n-1\}$ que es justamente $\mathbb{Z}_n^*$, y como sabemos, para saber la cantidad de elementos existe la función $\phi(n)$, que nos dice la cantidad de elementos coprimos con $n$ en $\mathbb{Z}_n$ o bien, la cantidad de elementos en $\mathbb{Z}_n^*$.

    De este modo, para $n = p\cdot q$ se tiene que $\phi(n) = (p-1)(q-1).$


    \item \textbf{Generalice el resultado anterior para {\boldmath$n$} entero positivo cualquiera.}

    A parrtir de lo visto en la parte anterior, el problema se reduce a encontrar la cantidad de elementos de $\mathbb{Z}_n^*$. Para esto existe la función $\phi$ que realiza justamente eso, sin embargo, explícitamente, se tiene que, para cualquier entero positivo $n$, siendo este el orden del grupo, la cantidad de elementos corresponde a
    $$\phi(n) = \prod_{i=i}^{k}(p_i^{e_i-1} (p_i - 1)).$$

    Siendo cada $p$ un primo de la descomposición prima de $n$ y cada $e$ la cantidad de veces que está presente en la misma.

\end{enumerate}

\section*{Problema 3}

\textbf{Construya PPT {\boldmath$\mathcal{A}'$} tal que la probabilidad de adivinar sea 0.99 para todo {\boldmath$x$}}.

Primero genero el siguiente algoritmo $\mathcal{A}^*(e, N, z)$ para conseguir lo deseado:

\begin{itemize}
    \item generar un $y \getunif \mathbb{Z}_N^*$
    \item computar $x^* = z \cdot y^e \mod N$.
    \item obtener $candidate = \mathcal{A}(e, N, x^*)$, de modo que si estoy en ese $0.01$ de probabilidad, lo que va a retornar el adversario es $x\cdot y$ ya que con un input de $(xy)^e$, retorna $xy$ en esa fracción de los casos.
    \item computar $trick = y^{-1} \cdot candidate$, ya que si $candidate$ es $xy$ entonces $trick = x$ ya que $y$ con su inversa se cancelan ($y$ tiene inversa ya que es parte de $\mathbb{Z}_N^*$ y es una propiedad de dicho conjunto).
    \item si $[trick^e \mod N] = z$ retornar $trick$, en otro caso retornar $null$.
\end{itemize}

El \textit{truco} está en utilizar la inversa de $y$ para que luego se cancele en el caso en el que ocurra lo deseado.

Ahora, utilizamos el algoritmo $\mathcal{A}^*$ para generar el algoritmo pedido, esto porque necesitamos que tenga éxito en los casos en los que el otro algoritmo falla, esto es, con una probabilidad de $0.99$.

Este algoritmo $\mathcal{A}'(e, N, z)$ itera una cantidad fija de veces $M$ que discutiré más adelante, y es así

\begin{itemize}
    \item para $i$ desde 0 a $M$
    \item si $\mathcal{A}^*(e, N, z) != \texttt{null}$ retornarlo
\end{itemize}

El análisis probabilístico es el siguiente, para encontrar $x$ al correr el algoritmo original, la probabilidad es $0.01$. Ahora, la probabilidad de que se encuentre $x$ en el $i$-ésimo paso del algoritmo corrresponde a $(0.01)(0.99)^{i-1}$ ya que tiene que haber caído en el caso contrario $i$ veces. Así, lo que se necesita obtener es un $i$ de modo que la probabilidad sea mayor o igual a $0.99$, esto es que se haya encontrado $x$ en $i$ pasos o menos, es decir
$$\sum_{k=1}{i}(0.01)(0.99)^{k-1} \geq 0.99$$

Esa sumatoria es equivalente a
$$1-(0.99)^i \geq 0.99$$

Y despejando $i$ se tiene que $i=458.2$, por lo que para que la probabilidad sea $0.99$ para todo $x$, se requiere que $M$ sea 459, y con esto el algoritmo es PPT, ya que todos los computos son polinomiales ya que aparte de cálculos triviales, se basa en el otro algoritmo que lo es y en computar la inversa de $y$ que viene dado por $\mathbb{Z}_n^*$.


\section*{Problema 4}

\textbf{Como un adversario puede escuchar la conversación bajo protocolo Diffie-Hellman y qué más puede hacer}.

El protocolo Diffie-Hellman se basa en generar un \textit{secreto común}, esto se logra compartiendo por un lado las características del grupo, que son $g$, $q$, $\mathbb{G}$ y además $g^\alpha$ donde $\alpha$ es la llave de Alice. Y por el otro devolviendo $g^\beta$ donde es análogo a $\alpha$ pero de Bob.

Así, si ambos computan $k = (g^\alpha)^\beta$ y $k = (g^\beta)^\alpha$ respectivamente, tendrán entonces un \textit{secreto común}. El problema, y lo que puede hacer el adversario, es ubicarse en el medio del canal de conversación y entregar a cada miembro de la misma un secreto distinto, de modo que el adversario establece una comunicación bajo Diffie-Hellman con cada uno de los participantes.

Más específicamente, lo que puede hacer es lo siguiente:

\begin{itemize}
    \item Alice envia hacia Bob $(g, q, \mathbb{G}, g^\alpha)$
    \item El adversario intercepta, guarda todo, genera una llave nueva $\gamma$ y se queda con $g^{\alpha\gamma}$.
    \item El adversario envia a Bob $(g, q, \mathbb{G}, g^\gamma)$
    \item Bob envía de vuelta hacia Alice $(g^\beta)$ y guarda $g^{\beta\gamma}$ pensando que es el \textit{secreto común} con Alice
    \item El adversario intercepta nuevamente, guarda $g^{\beta\gamma}$
    \item El adversarrio envia a Alice $(g^\gamma)$
    \item Alice guarda $g^{\alpha\gamma}$ pensando que es el secreto común con Bob
\end{itemize}

Así el adversario tiene un secreto común con Alice: $g^{\alpha\gamma}$, y a su vez uno con Bob: $g^{\beta\gamma}$, sin embargo, Alice y Bob no tienen un secreto común. Esto permite al adversario controlar absolutamente toda la conversación, ya que puede 1) escuchar lo que se mandan y transmitirlo intacto, 2) escuchar lo que se mandan y modificarlo, 3) enviar cosas sin que ninguno de los dos haya mandado algo (posterior al intercambio de llaves). Hago la última distinción porque el caso 2) corresponde a cambiar mensajes, pero la intención de enviarlos existía, en el caso 3) puede enviarlos sin que haya intención de ninguno de los 2 participantes de mandar algo. En el fondo, el adversario impersona a Alice y a Bob cuando estime conveniente.


\end{document}
